\documentclass{article}
\usepackage {inputenc, fullpage, listings, amsmath}

\parindent 0pt

\title{%
   CSc 360: Operating Systems (Spring 2022) \\
   \large Written Assignment 1 (W1) \\
    Alex Holland V00}
    
\date{}

\begin{document}

\maketitle

1.\\
(a)
\begin{itemize}
  \item User mode is not given access to the memory or hardware.
  \item If the User mode wants to access system resources, it must use a system API.
  \item The kernel mode is the privileged mode where the process is given unrestricted access to the memory or hardware.
  \item Crashes that occur in Kernel mode are much more serious then User mode.
\end{itemize}

(b)
\begin{itemize}
    \item Kernel mode is reserved for the most trusted, low level operating system instructions.
    \item In User mode, the operating system can prevent many crashes since a crash should only affect the running user program.
    \item A User mode can also prevents a users program from 'accidentally' overwriting the operating system with user data.
\end{itemize}

(c)
\begin{itemize}
    \item Mode switch refers to the process of switching the bit from 1 for user mode to 0 for kernel mode, and vice versa. This happens through the use of system calls.
    \item Context switch refers to the process of storing a process or a thread such that it can be resumed execution at a later time.
    \item Context switching happens only in Kernel mode.
\end{itemize}

(d)\\
{\bf Pros:}
\begin{itemize}
    \item High security and reliability due to most services running in user mode.
    \item When compared to a monolithic kernel, it is easier to port from one hardware architecture to another. 
    \item It is easy to extend the operating system, since new services are created in the user space and do not require modification of the kernel.
    \item Architecture is smaller then a monolithic kernel.
\end{itemize}
{\bf Cons:}
\begin{itemize}
    \item Sometimes slow performance due to high system-function overhead.
    \item Overhead often involved when switching between processes and copying messages.
\end{itemize}

\bigskip

2.\\
(a)
\begin{align*}
Output\;1: && Output\;2: && Output\;3:\\
0 && 0 && 0\\
2 && 1 && 2\\
1 && 2 
\end{align*}
*Note: $Output \; 3$ is the result if fork() fails.\\

\smallskip
(b)
\begin{lstlisting}[mathescape=true]
#define OUTPUT printf("%d\n", i)

main() {
    int i = 0; OUTPUT;

    if (fork()) {
        $\bf{wait(NULL);}$
        i += 2; OUTPUT;
    } else {
        i += 1; OUTPUT; return(0);
    }
}
\end{lstlisting}

\bigskip

3.\\
(a)\\ 
Feasible: A process can go from running to blocked when wait() is called or input is received.\\

\smallskip
(b)\\
Not feasible: A process can not go from a blocked to a running state. The process must be in the ready state before it can be in the running state.\\

\smallskip
(c)\\
Feasible: A process can go from a blocked to ready state if I/O or event is completed whilst in the blocked state.\\

\smallskip
(d)\\
Not feasible: A process in the ready state cannot be blocked because it does not have access to the CPU. The process cannot be blocked if it has not started execution.\\

\smallskip
(e)\\
Feasible: A process can go from a ready to running state when the scheduler dispatcher chooses the process for execution.\\

\smallskip
(f)\\
Feasible: A process can go from a running to ready state when an interrupt is thrown, which pauses the current process execution.\\

\end{document}
